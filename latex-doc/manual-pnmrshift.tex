\documentclass[11pt]{report}

\usepackage{stix}                                                                
%                                                                                
% fontenc, mtpro/mathtime are for commercial Math fonts. Can be commented out    
%\usepackage[LY1]{fontenc} %% needed for YandY TeX but works with MiKTeX also    
%\usepackage[subscriptcorrection,slantedGreek,nofontinfo,lucidacal,mtpbbi]{mtpro2}
                                                                                 
% proper URL formatting and hyperlink embedding:                                 
%\usepackage[implicit=false,hidelinks]{hyperref}                                  
                                                                                 
%%% AMS Math                                                                     
\usepackage{amsmath}                                                             
                                                                                 
%%% Bold math definitions                                                        
\usepackage{bm}                                                                  
%\usepackage{relsize}                                                            
%%% graphics import support                                                      
\usepackage{graphicx}                                                            
\graphicspath{ {./figures/} }                                                    
%%% Misc. Packages                                                               
                                                                                 
\usepackage{overcite}                                                            
\usepackage{setspace}                                                            
\usepackage{color}                                                               
\usepackage{rotating}                                                            
\usepackage[version=3]{mhchem}                                                              

% These should be loaded before the font in this case
%\RequirePackage{amssymb}
%%%%%%%%%%%%%%%%%%%%%%%%%%%%%%%%%%%%%%%%%%%%%%%%%%%%%%%%%%%%%%%%%%

%%%%%%%%%%%%%%%%%%%%%%%%%%%%%%%%%%%%%%%%%%%%%%%%%%%%%%%%%%%%%%%%%%%
% Font
%%%%%%%%%%%%%%%%%%%%%%%%%%%%%%%%%%%%%%%%%%%%%%%%%%%%%%%%%%%%%%%%%%%
%\RequirePackage{kpfonts}  % Will load amsmath
%\RequirePackage[T1]{fontenc}
%\RequirePackage{microtype}
\linespread{1.05}
%\setlength{\jot}{3.0ex}
%%%%%%%%%%%%%%%%%%%%%%%%%%%%%%%%%%%%%%%%%%%%%%%%%%%%%%%%%%%%%%%%%%%

%\RequirePackage[sfdefault=iwona]{isomath}
\renewcommand\bmmax{2}
\renewcommand\hmmax{0}
\RequirePackage{bm}


\usepackage{import}


\usepackage{xspace}

\RequirePackage{geometry}
\geometry{lmargin=1.25in,rmargin=1.25in,vmargin=1.00in}

\usepackage{listings}
\lstset{language={}}
\lstset{frame=single,basicstyle=\ttfamily\small,columns=fixed,showspaces=false}
\renewcommand\lstlistlistingname{List of Listings}


%\RequirePackage[super,sort&compress,comma]{natbib}
%\RequirePackage{natmove}

%\renewcommand\bibname{References}


% Macros, etc
\newcommand\PNMRShift{\textsf{PNMRShift}\xspace}

\newcommand{\code}[1]{\lstinline!#1!}
\newcommand{\etal}{\mbox{\emph{et al}}\xspace}
\newcommand{\matdim}[2]{\ensuremath{#1 \times #2}}
\renewcommand{\vec}{\bm}
\newcommand{\mat}{\bm}    % Matrices are bold
%\newcommand{\ten}{\mathsf}
\newcommand{\ten}{\bm}

\newcommand{\sci}[2]{\ensuremath{#1\times10^{#2}}}
\newcommand{\Order}{\ensuremath{\mathscr{O}}}  % Order symbol

\newcommand{\vecop}[1]{\op{\vec{#1}}}
\newcommand{\matop}[1]{\op{\mat{#1}}}
\newcommand{\tenop}[1]{\op{\ten{#1}}}
\newcommand{\scop}[1]{\ensuremath{\hat{#1}}} %% scalar operators
%\newcommand{\scop}[1]{\op{#1}}

\newcommand{\transpose}{\mathsf{T}}
\newcommand{\conj}{*}

\renewcommand{\Re}{\operatorname{Re}}


\newcommand\nr{\mathrm{nr}}
\newcommand\so{\mathrm{so}}

\newcommand\nrfc{\mathrm{nrfc}}
\newcommand\nrsd{\mathrm{nrsd}}

\newcommand\sofc{\mathrm{sofc}}
\newcommand\sosd{\mathrm{sosd}}
\newcommand\soas{\mathrm{as}}

\newcommand\fc{\mathrm{fc}}
\newcommand\pc{\mathrm{pc}}
\newcommand\sd{\mathrm{sd}}

\newcommand\iso{\mathrm{iso}}
\newcommand\dia{\mathrm{dia}}
\newcommand\para{\mathrm{para}}
\newcommand\orb{\mathrm{orb}}
%%%%%%%%%%%%%%%%%%%%%%%%%%%%%%%%%%%%%%%%%%%%%%%%%%%%%%%%%%%%%%%%%%%%%%%%%%%%%%%%
%%%%%%%%%%%%%%%%%%%%%%%%%%%%%%%%%%%%%%%%%%%%%%%%%%%%%%%%%%%%%%%%%%%%%%%%%%%%%%%%

\begin{document}

\begin{titlepage}
\begin{center}
{\Large \textbf{PNMRShift Program}}


\vspace*{1cm}

{\large
Benjamin Pritchard, Bob Martin, and Jochen Autschbach\\[2ex]
Department of Chemistry\\
University at Buffalo, State University of New York\\
Buffalo, NY 14260-3000, USA\\[2ex]
email: jochena@buffalo.edu\\[3ex]
last compiled: \today     
}
\end{center}  
\end{titlepage}

%\listoftodos
\tableofcontents
%\listoffigures
%\listoftables
%\listofschemes
%\listofexamples
%\lstlistoflistings
%%%%%%%%%%%%%%%%%%%%%%%%%%%%%%%%%%%%%%%%%%%%%%%%%%%%%%%%%%%%%%%%%%%%%%%%%%%%%%%%



%\subimport{chap-intro/}{chap-intro}
%\subimport{chap-usage/}{chap-usage}
%\subimport{chap-examples/}{chap-examples}


\chapter{Introduction}
\label{chap:intro}

The \PNMRShift program calculates the NMR shielding for a nucleus in a
paramagnetic molecule using EPR spin Hamiltonian parameters for the
shielding components specific to the electron paramagnetism. The
default route of the code is based on an equation derived by Moon \&
Patchkovskii (Reference~\citen{Patchkovskii:2003a}):
%
\begin{equation}
    \ten{\sigma} = \ten{\sigma}^{\mathrm{orb}} - \frac{\beta_e}{g_N \beta_N}\frac{S(S+1)}{3k_{B}T} \ten{g} \cdot \ten{A}
\label{eqn:pnmrtotal}
\end{equation}
%
with $\ten{A}$ in energy units.
In SI units, $\ten{A}(\mathrm{J}) = \sci{h}{6} \times \ten{A}(\mathrm{MHz})$.
The result of this equation is unitless, but usually
multiplied by $10^{6}$ to convert it to the usual ppm units. The dot
indicates here a matrix product. $S$ is the effective spin, and other
physical constants have usual notations. The spin Hamiltonian is assumed to
have the following contributions: $\beta_e \vec{B}\, \ten{g} \,
\vec{S}$ for the Zeeman interaction, and $\vec{S}\, \ten{A}\, \vec{I}$
for the hyperfine interaction. If the hyperfine interaction is based
on the operator $\vec{I}\, \ten{A}\, \vec{S}$, for instance, then a
matrix transpose for $\ten{A}$ should be used in Equation
\ref{eqn:pnmrtotal}. 

In the simplest case, \PNMRShift reads from an input file calculated
tensors $\ten{g}$, $\ten{A}$ and $\ten{\sigma}^{\mathrm{orb}}$ (the
latter is usually the corresponding shielding tensor of a closely
related diamagnetic analog, or calculated for the paramagnetic system
but excluding any contributions related to the electron
paramagnetism), from the command line or a batch script used to call
the program the values for $S$ and $T$, a molecular structure from an
xyz file, and assembles the shielding tensor as in Equation
\ref{eqn:pnmrtotal}, or more sophisticated versions as explained
below. For worked-out examples that we have used for an undergraduate
computational chemistry laboratory course please see a related article in
\textit{J.\ Chem.\ Educ.} \cite{Autschbach:2013h}. 

Equation~\ref{eqn:pnmrtotal} can be split up into several components
in order to facilitate analysis. The $g$-tensor is divided into three
parts -- the first is the nonrelativistic isotropic part due to $g_e$,
and the remaining (relativistic) part,
$\ten{\Delta g} = \ten{g} - g_e\mat{1}$, is further subdivided into
its own isotropic part $\Delta g_\iso \mat{1}$, and its anisotropic
part $\Delta \tilde{g}$.
%
\begin{align}
       \ten{g} &= g_e\mat{1} + \ten{\Delta g} \\
\ten{\Delta g} &= \ten{g} - g_e\mat{1} = \Delta g_\iso\mat{1} + \ten{\Delta \tilde{g}}
\end{align}
%
Likewise, the hyperfine tensor is decomposed into five separate components -
a nonrelativistic Fermi-Contact term $\ten{A}_\nrfc$, nonrelativistic spin-dipole term $\ten{A}_\nrsd$,
relativistic corrections to the nonrelavistic terms ($\ten{A}_\sofc$ and $\ten{A}_\sosd$),
and a relativstic asymmetric part $\ten{A}_\soas$. The Fermi-Contact terms are often taken to be
diagonal, isotropic tensors.

Therefore, $\ten{g} \cdot \ten{A}$ in Eq.~\ref{eqn:pnmrtotal} is
a 15-term sum of these combinations.\cite{Pennanen:2008a,Liimatainen:2009a} 
The 15 terms, along with the notation used in the output of \PNMRShift,
are shown in Table~\ref{table:totalcontrib}.

\begin{table}
\centering
\caption{All contributions to $\ten{g} \cdot \ten{A}$ Eq.~\ref{eqn:pnmrtotal}}
\label{table:totalcontrib}
%\scalebox{1.15}{
\begin{tabular}{ccc}
\hline
Term & Label & Order \\ 
\hline
\multicolumn{3}{c}{Fermi-Contact Shift} \\
$g_e \ten{A}_\nrfc$ & \code{GE.ANRFC} & \Order($\alpha^2$) \\
$g_e \ten{A}_\sofc$ & \code{GE.ASOFC} & \Order($\alpha^4$) \\
$\Delta g_\iso \ten{A}_\nrfc$ & \code{dGiso.ANRFC} & \Order($\alpha^4$) \\
$\Delta g_\iso \ten{A}_\sofc$ & \code{dGiso.ASOFC} & \Order($\alpha^6$) \\
$\ten{\Delta \tilde{g}} \ten{A}_\nrfc$ & \code{dGtilde.ANRFC} & \Order($\alpha^4$) \\
$\ten{\Delta \tilde{g}} \ten{A}_\nrsd$ & \code{dGtilde.ANRSD} & \Order($\alpha^4$) \\[1.5ex]
\multicolumn{3}{c}{Pseudocontact Shift} \\
$g_e \ten{A}_\nrsd$ & \code{GE.ANRSD} & \Order($\alpha^2$) \\
$g_e \ten{A}_\sosd$ & \code{GE.ASOSD} & \Order($\alpha^4$) \\
$g_e \ten{A}_\soas$ & \code{GE.AS} & \Order($\alpha^4$) \\[1.5ex]
$\Delta g_\iso \ten{A}_\nrsd$ & \code{dGiso.ANRSD} & \Order($\alpha^4$) \\
$\Delta g_\iso \ten{A}_\sosd$ & \code{dGiso.ASOSD} & \Order($\alpha^6$) \\
$\Delta g_\iso \ten{A}_\soas$ & \code{dGiso.AS} & \Order($\alpha^6$) \\[1.5ex]
$\ten{\Delta \tilde{g}} \ten{A}_\sofc$ & \code{dGtilde.ASOFC} & \Order($\alpha^6$) \\
$\ten{\Delta \tilde{g}} \ten{A}_\sosd$ & \code{dGtilde.ASOSD} & \Order($\alpha^6$) \\
$\ten{\Delta \tilde{g}} \ten{A}_\soas$ & \code{dGtilde.AS} & \Order($\alpha^6$) \\[1.5ex]
\hline
\end{tabular}
%}
\end{table}

Equation~\ref{eqn:pnmrtotal} is also split up into terms associated with
the Fermi-Contact portions and the spin-dipole terms of the hyperfine tensor.
These are commonly called the contact or Fermi-Contact (FC) shift and the
pseudocontact (PC) shift, respectively.
%
\begin{gather}
\ten{\sigma} = \ten{\sigma}^{\mathrm{orb}}
               - \underbrace{K \ten{g} \cdot \ten{A}^\fc}_\mathrm{FC} 
               - \underbrace{K \ten{g} \cdot \ten{A}^\sd}_\mathrm{PC}\\
               K = \frac{\beta_e}{g_N \beta_N}\frac{S(S+1)}{3k_{B}T}
\end{gather}
%
Which of the 15 terms in the expansion belong to which contribution
is also shown in Table~\ref{table:totalcontrib}.
Under certain assumptions,
these equations are equivalent to those given by McConnell \& Robertson\cite{McConnell:1958b},
Kurland \& McGarvey\cite{Kurland:1970a}, and Bertini et al.\cite{Bertini:2002a}

\section{Including Zero-Field Splitting (ZFS)}
\label{sec:zfs}
\subsection{Method of Soncini and Van den Heuvel}

For molecules with spin $S> 1/2$ and calculations where the $g$- and
hyperfine tensors are determined without spin-orbit interactions
included in the ground state wavefunction or density matrix,
zero-field splitting should be included in the PNMR shielding
equation. The equation analogous to Eq.~\ref{eqn:pnmrtotal}
is\cite{Soncini:2013a,Autschbach:2014p}
%
\begin{align}
\label{eqn:pnmrzfstotal}
\ten{\sigma} = \ten{\sigma}^{\mathrm{orb}} - \frac{\beta_e}{g_N \beta_N k_{B} T}
\ten{g} \cdot   \ten{Z} \cdot \ten{A}
\end{align}
%
The elements of the matrix $\ten{Z}$ are given by
%
\begin{align}
\label{eqn:sonc-sso}
Z_{kl}=\frac{1}{Q}
\sum_{\lambda} & e^{-                             
\frac{E_\lambda}{k_B T}} \left[\sum_{a,a'}\langle S\lambda a\vert \scop{S}_{k}\vert
S\lambda a'\rangle \langle S                                                     
\lambda a'\vert \scop{S}_{l}\vert S\lambda a\rangle \right. \nonumber \\               
&\left. +2 k_B T \; \Re \sum_{\lambda'\neq\lambda}                                         
\sum_{a,a'}\frac{\langle S\lambda a\vert \scop{S}_{k}\vert S\lambda'a'\rangle \langle  
S\lambda'a'\vert \scop{S}_{l}\vert S\lambda a\rangle}{E_{\lambda'}-E_{\lambda}} \right]
\end{align}
%
The energies $E_{\lambda}$ are obtained from diagonalizing the
$\vec{S}\mat{D}\vec{S}$ part of the EPR Hamiltonian where $\mat{D}$ is
the ZFS tensor.  The labels $\lambda$ and $a$ refer to the $2S+1$ ZFS
eigenfunctions, the eigenvectors of the $\vec{S}\mat{D}\vec{S}$ term,
which are not necessarily degenerate as they are for a pure spin
multiplet without ZFS. $Q$ is the partition function for the split
multiplet.  For more information, see
References~\citen{Soncini:2013a}, ~\citen{Autschbach:2014p}, and
~\citen{Soncini:2012a}.
%
\subsection{Method of Pennanen and Vaara}
Previous versions of \PNMRShift accounted for ZFS by calculating the PNMR
shielding using the method of Vaara et al.\ 
\cite{Pennanen:2008a,Liimatainen:2009a}, who used
%
\begin{align}
\label{eqn:vaara-sso}
Z_{kl}=\frac{1}{Q}
\sum_{\lambda} & e^{-                             
\frac{E_\lambda}{k_B T}} \left[\sum_{a}\langle S\lambda a\vert
                 \scop{S}_{k}\, \scop{S}_{l}\vert S\lambda a\rangle
                 \right ]
\end{align}
%
after diagonalizing the ZFS Hamiltonian in the basis of pseudo-spin-$S$
eigenfunctions. This method is deprecated as it was shown by Soncini
and Van den Heuvel to be incorrect. 


\section{Spin-orbit correction for the Hyperfine Tensor}
\label{sec:hypcorrect}

If the hyperfine tensor is calculated without considering spin-orbit
effects, there are certain approximations to the PNMR shift equations
that are not entirely compatible with each other, as discussed in
Reference \citen{Autschbach:2011c}. This is due to the use of $g_e$
versus the full $g$-tensor in the equations. For organic radicals and
perhaps first-row transition metals (generally, when the spin-orbit
splitting is small compared to the gap between ground state and
excited electronic states), these considerations should not matter
much. For cases with stronger spin-orbit coupling, we proposed a
correction to the hyperfine tensor that uses the calculated $g$-tensor
elements \cite{Autschbach:2011c}. The correction should be used with
caution, however, as the full implications of applying it are not yet
fully understood. 


% \section{Dipolar Hyperfine Tensor}
% \label{sec:diphyp}
% The spin-dipole part of the hyperfine tensor ($\ten{A}^\sd$) can be
% estimated using just the geometry. This assumes that the magnetization
% can be approximated by a point magnetic dipole.  In general, this
% equation is given as
% %
% \begin{gather}
% \ten{A}^\sd = \frac{\mu_0}{4\pi r^5} g_e \beta_e g_N \beta_N 
%                           \left [ 3\vec{r}\otimes\vec{r} - r^2\mat{1} \right ]
%                           \label{eqn:simpledipatens}
% \end{gather}
% %
% where the vector $\vec{r}$ is a vector from the paramagnetic center to the nucleus for which the tensor
% is being calculated. The paramagnetic center is often placed at the origin of the coordinate system,
% in which case $\vec{r}$ is conveniently the cartesian coordinate of the nucleus. This tensor
% is traceless, and therefore can only contribute to the pseudocontact shift.
% Eq.~\ref{eqn:simpledipatens} is often rewritten, giving element $ab$ of the hyperfine tensor
% %
% \begin{equation}
%   \ten{A}_{ab}^\sd = \frac{\mu_0}{4\pi} g_e\beta_e g_N \beta_N 
%                                  \left\langle \frac{3ab}{r^5} - \frac{\delta_{ab}}{r^3} \right\rangle \label{eqn:simpledipatenssuccint}
% \end{equation}


% As seen in Section~\ref{sec:hypcorrect}, this tensor can be corrected, forming
% %
% \begin{gather}
% \ten{A}^\sd = \ten{T}^\sd \cdot \ten{g} \label{eqn:simpledipatenscorrect} \\
% \ten{T}^\sd = \frac{\mu_0}{4\pi r^5} \beta_e g_N \beta_N \left [ 3\vec{r}\otimes\vec{r} - r^2\mat{1} \right ]
% \end{gather}
% %
% or, in an alternate way,
% %
% \begin{gather}
% \ten{A}^\sd_{ab} = \sum_{i\in\left\{x,y,z\right\}} \ten{T}^\sd_{bi} \ten{g}_{ia}  \label{eqn:simpledipatenscorrectelement}
% \end{gather}
% %
% Using just this corrected, purely-dipolar hyperfine tensor in the overall PNMR shift equation (Eq.~\ref{eqn:pnmrtotal})
% results in an expression that is identical to the pseudocontact
% expression given by Bertini \etal.


\chapter{Compilation and Usage}
\label{chap:usage}

\section{Compilation}
\subsection{Requirements}
Compiling and using the \PNMRShift program requires a Unix like system
(Linux et al.) with a working set of compilers:
%
\begin{itemize}
\item A Fortran and C++ Compiler. Tested with:
\subitem GCC v4.4.6
\subitem GCC v4.7.2
\subitem Intel v12.1
\item Boost (no compilation needed, just the Boost::format headers)
\item CMake 2.6 or newer 
\end{itemize}
%
The code is fairly standard C++ code, therefore other compilers should
work just as well.


\subsection{Compiling}
Compilation is performed on the command line using CMake 
%
\begin{lstlisting}[language=Bash]
mkdir build
cd build
cmake ../
make
\end{lstlisting}
%
And that is it.
After building, the PNMRShift executable may be copied to anywhere
you find convenient and where it is accessible. The executable
is all that is needed.

\subsection{Windows executables}

 See the file \texttt{README.windows} in the top
level directory of the \PNMRShift package for instructions how to
cross-compile a Windows executable on a Linux system. For
convenience, a pre-compiled Windows binary is provided. It needs
to be executed from a command line window. We recommend using batch
files (\texttt{.bat}) to run the code, as in the provided examples. 

\section{Usage}
\subsection{Basic Usage}

If the program is invoked without options, a help screen is displayed. Table
\ref{table:cmdline} shows the options for the program.

\begin{table}
\centering
\caption{Arguments to the \PNMRShift program}
\label{table:cmdline}
\begin{tabular}{lp{3in}}
\hline
\textbf{Required}       & \\
\code{-t <temperature>} & Temperature in K \\
\code{-s <spin>}        & Overall spin of the system. Doublet = 0.5, triplet = 1.0, etc. \\ 
\code{-c <file>}        & XYZ file. This is a standard XYZ file with the first line being
                          the number of atoms and the second line being the title (or a
                          blank line). XYZ coordinates are in angstroms \\
\code{-f <file>}        & Input file for the program containing the tensor information \\
\\
\textbf{Optional}       & \\
\code{--geosd}          & Use a purely-dipolar hyperfine tensor that is generated solely from the geometry. \\
\code{--keepfc}         & When combined with --geosd, keep the Fermi-Contact portions of the hyperfine tensor. Default is to erase them.\\
\code{--correctA}       & Correct the hyperfine tensor with the $g$-tensor. See Section~\ref{sec:hypcorrect} \\
\code{--detail}         & Print the contributions to the overall shielding tensor of all 15 terms in Table~\ref{table:totalcontrib}. (Page~\pageref{table:totalcontrib})\\ 
\code{-a <label>:<avg>} & Averaging. See Section~\ref{sec:averaging}. Averaging can also be included in the input file.\\
\code{--deltag}         & Input file contains delta-g (in ppt) rather than the full g-tensor.\\
\code{--splitinp}       & Input file contains tensors split into components (g-tensor into dia- and paramagnetic, hyperfine into FCSD and PSOSO). Implies \code{--deltag}.\\
\code{--pvzfs}          & Calculates the ZFS correction using the method of Pennanen and Vaara. Default is to use the method of Soncini and Van den Heuvel (See Section~\ref{sec:zfs})\\
\hline
\end{tabular}
\end{table}


\subsection{Selection Strings}
\label{sec:selection}
In some places atoms can be specified with a string of numbers that
includes ranges. For example, for all atoms 1 through 10,
a string of \code{1-10} can be written. The following are all
valid selection strings:
%
\begin{itemize}
\item[] \code{1,2,3,4,5}
\item[] \code{1-5}
\item[] \code{2,5,6-10,22}
\item[] \code{1-3,6,10-18,21,49-65}
\end{itemize}
%
We hope that their use is self explanatory. Take note that
duplicates will be removed and the list will sorted. That is,
\code{1-4,2,6,12} will include nuclei 1, 2, 3, 4, 6, and 12.


\subsection{\PNMRShift Input File}
\label{sec:pnmrinp}
\PNMRShift
needs an input file containing the hyperfine tensors,
$g$-tensors, and orbital shielding tensors, optionally ZFS tensors,
as well as information on how to average certain quantities.

The file is constructed in a somewhat free form.  All keywords are
case insensitive, and numbers on the same line are separated with any
number of spaces or tabs.  (The geometry is read from a separate XYZ
file.)  The tensors are listed in any order.  Each tensor is started
with a header identifying which tensor it is, with the atom number
corresponding to the atom ordering in the XYZ file) which is then
followed by the tensor. The tensor itself is written in three or six
lines containing three columns each. Six lines are used if the tensors
are separated into contributions, which is needed for the detailed
analysis according to Table \ref{table:totalcontrib}.

A hyperfine tensor is written as a 3x3 matrix (three lines) in
units of MHz.
The header line
is  \code{atensor # X} where \code{X} is the element symbol
and \code{#} is the number of the atom in the XYZ input ordering,
with the first element being one.
The file can contain multiple hyperfine tensor specifications.
%
\begin{lstlisting}[language={}]
atensor 2 H
-6.565  -2.565  -1.098
-2.565  -0.262   3.088
-1.098   3.088  -3.199
\end{lstlisting}

A similar input is used for the orbital shielding tensor,
except the heading is now \code{orbtensor}.
The input of the shielding tensor is in ppm.
%
\begin{lstlisting}[language={}]
orbtensor 2 C
75.872  17.920   5.629
17.920  31.850 -15.832
 5.629 -15.832 164.274
\end{lstlisting}

The $g$-tensor is input with the keyword \code{gtensor}
with nothing else on the line.
%
\begin{lstlisting}[language={}]
gtensor
2.085     -0.000      0.000
0.000      2.085      0.000
0.000      0.000      2.003
\end{lstlisting}

Finally, the optional ZFS tensor is input under the heading
\code{zfstensor}
%
\begin{lstlisting}[language={}]
zfstensor
 0.659659     -0.186353     -0.907466
-0.186353     -0.629593      0.128537
-0.907466      0.128537     -0.030066
\end{lstlisting}

After construction, this file is used
as an input to \code{-A}, \code{-O}, \code{G}, and/or
\code{-Z} arguments to the program.


\subsection{Averaging}
\label{sec:averaging}
The \PNMRShift program allows the user to average nuclei. This is particularly
useful for symmetry-equivalent nuclei whose calculated shifts are not equal
due to the calculation employing a broken-symmetry configuration,
which is not uncommon with DFT calculations.

To specify an average, use the \code{-a} argument, followed by a
label and selection string (Section~\ref{sec:selection}). The label
is an arbitrary name given (for example, "Methyl 1H"). The
label and selection string are separated by a colon. Usually, the
entire argument is surrounded in quotes to properly handle spaces
in the label. For example, averaging three methyl groups, both
carbon and proton shieldings, may look like
%
\begin{itemize}
\item[] \code{-a "Methyl 1H:8-10,19-21,23-25" -a "Methyl 13C:7,18,22"}
\end{itemize}
%
Averaging information can also be included in the input file, under the heading
'averaging'. Each different average is placed on its own line. The section
is processed until a blank line occurs.

\begin{lstlisting}[language={}]
averaging
ethyl 1H:8-10,19-21,23-25
ethyl 13C:11,22,26
\end{lstlisting}

In the output, the average will be printed in the main (isotropic)
results table, as well as in separate detailed results tables.

\section{Output}
\label{sec:output}
The output of \PNMRShift is structured as follows (in order):
%
\begin{enumerate}
\item{The user input (temperature, spin, and atomic coordinates in \AA)}
\item{The $g$-tensor components, as well as its conversion from $\Delta g$ to $g$}
\item{Hyperfine tensors as read in from the input file}
\item{Hyperfine tensors that are corrected or calculated using the geometry (\code{--correctA} and/or \code{--geosd})}
\item{Orbital shieldings as read from the input file}
\item{ZFS Tensor. If there was no ZFS tensor input, this will be a \matdim{3}{3} identity matrix}
\item{Calculated $\ten{Z}$ matrix (see Eq.~\ref{eqn:sonc-sso}). If no ZFS tensor was input, this
will be a \matdim{3}{3} diagonal matrix with elements of $S(S+1)/3$}
\item{Main results table, showing the predicted isotropic values of the shielding, Fermi-Contact, and pesudocontact
tensors)}
\item{If requested, a detailed breakdown of contributions with full tensors}

\end{enumerate}

If an atom was not input in some section (for example, if the input is
missing the hyperfine tensor for an atom), that atom will show an `xx'
in the element column and the tensor may be zero.

\chapter*{Acknowledgments}

The development of the \PNMRShift program was made possible with
financial support from the U.S.\ Department of Energy, Office of Basic
Energy Sciences, Heavy Element Chemistry program, under grant
DE-FG02-09ER16066. The views, opinions, and facts stated in this
document are those of the authors, not of the DOE or anybody else. We
thank Dr. Pennanen for providing us with a binary of his PNMR program
and an example input for \ce{O2} which we used to verify our
implementation of Equation \ref{eqn:vaara-sso} and the breakdown into
the contribution listed in Table \ref{table:totalcontrib}.

%\bibliographystyle{myachemso}
% \bibliography{myjournals,autschbach,actinides,references,cyanine,dft,dft-nmr,pnmr,nmr-exp,nmr-refs,excitations,cdrefs,basis,relativity,response,solvent,zfs,misc}
%\bibliography{thecitations}

\providecommand{\refin}[1]{\\ \textbf{Referenced in:} #1}
\begin{thebibliography}{10}

\bibitem{Patchkovskii:2003a}
Moon,~S.;\ \ Patchkovskii,~S.  First--principles calculations of paramagnetic
  {NMR} shifts.   In  \textit{Calculation of {NMR} and {EPR} {P}arameters.
  {T}heory and {A}pplications}; Kaupp,~M.;\ \ B{\"u}hl,~M.;\ \ Malkin,~V.~G.,\
  \ Eds.;  Wiley-VCH: Weinheim, 2004 pages 325--338.

\bibitem{Autschbach:2013h}
Pritchard,~B.;\ \ Simpson,~S.;\ \ Zurek,~E.;\ \ Autschbach,~J. \textit{J.\
  Chem.\ Educ.} \textbf{2014,} \textsl{91,} 1058-1063.

\bibitem{Pennanen:2008a}
Pennanen,~T.~O.;\ \ Vaara,~J. \textit{Phys. Rev. Lett.} \textbf{2008,}
  \textsl{100,} 133002--4.

\bibitem{Liimatainen:2009a}
Liimatainen,~H.;\ \ Pennanen,~T.~O.;\ \ Vaara,~J. \textit{Can. J. Chem.}
  \textbf{2009,} \textsl{87,} 954--964.

\bibitem{McConnell:1958b}
McConnell,~H.~M.;\ \ Robertson,~R.~E. \textit{J. Chem. Phys.} \textbf{1958,}
  \textsl{29,} 1361--1365.

\bibitem{Kurland:1970a}
Kurland,~R.~J.;\ \ McGarvey,~B.~R. \textit{J. Magn. Reson.} \textbf{1970,}
  \textsl{2,} 286--301.

\bibitem{Bertini:2002a}
Bertini,~I.;\ \ Luchinat,~C.;\ \ Parigi,~G. \textit{Prog. Nucl. Mag. Res. Sp.}
  \textbf{2002,} \textsl{40,} 249--273.

\bibitem{Soncini:2013a}
Soncini,~A.;\ \ Van~den Heuvel,~W. \textit{J.\ Chem.\ Phys.} \textbf{2013,}
  \textsl{138,} 021103.

\bibitem{Autschbach:2014p}
Martin,~B.;\ \ Autschbach,~J. \textit{J.\ Chem.\ Phys.} \textbf{2015,}
  \textsl{142,} 054108.

\bibitem{Soncini:2012a}
Van~den Heuvel,~W.;\ \ Soncini,~A. \textit{Phys.\ Rev.\ Lett.} \textbf{2012,}
  \textsl{109,} 073001.

\bibitem{Autschbach:2011c}
Autschbach,~J.;\ \ Patchkovskii,~S.;\ \ Pritchard,~B. \textit{J.\ Chem.\ Theory
  Comput.} \textbf{2011,} \textsl{7,} 2175-2188.

\end{thebibliography}




\end{document}

%%% Local Variables:
%%% mode: latex
%%% TeX-master: t
%%% End:
